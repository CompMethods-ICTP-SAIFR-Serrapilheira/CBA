% Options for packages loaded elsewhere
\PassOptionsToPackage{unicode}{hyperref}
\PassOptionsToPackage{hyphens}{url}
%
\documentclass[
]{article}
\usepackage{amsmath,amssymb}
\usepackage{lmodern}
\usepackage{iftex}
\ifPDFTeX
  \usepackage[T1]{fontenc}
  \usepackage[utf8]{inputenc}
  \usepackage{textcomp} % provide euro and other symbols
\else % if luatex or xetex
  \usepackage{unicode-math}
  \defaultfontfeatures{Scale=MatchLowercase}
  \defaultfontfeatures[\rmfamily]{Ligatures=TeX,Scale=1}
\fi
% Use upquote if available, for straight quotes in verbatim environments
\IfFileExists{upquote.sty}{\usepackage{upquote}}{}
\IfFileExists{microtype.sty}{% use microtype if available
  \usepackage[]{microtype}
  \UseMicrotypeSet[protrusion]{basicmath} % disable protrusion for tt fonts
}{}
\makeatletter
\@ifundefined{KOMAClassName}{% if non-KOMA class
  \IfFileExists{parskip.sty}{%
    \usepackage{parskip}
  }{% else
    \setlength{\parindent}{0pt}
    \setlength{\parskip}{6pt plus 2pt minus 1pt}}
}{% if KOMA class
  \KOMAoptions{parskip=half}}
\makeatother
\usepackage{xcolor}
\IfFileExists{xurl.sty}{\usepackage{xurl}}{} % add URL line breaks if available
\IfFileExists{bookmark.sty}{\usepackage{bookmark}}{\usepackage{hyperref}}
\hypersetup{
  pdftitle={Final project report - computational methods},
  pdfauthor={Amanda Costa Ayres Salmeron},
  hidelinks,
  pdfcreator={LaTeX via pandoc}}
\urlstyle{same} % disable monospaced font for URLs
\usepackage[margin=1in]{geometry}
\usepackage{graphicx}
\makeatletter
\def\maxwidth{\ifdim\Gin@nat@width>\linewidth\linewidth\else\Gin@nat@width\fi}
\def\maxheight{\ifdim\Gin@nat@height>\textheight\textheight\else\Gin@nat@height\fi}
\makeatother
% Scale images if necessary, so that they will not overflow the page
% margins by default, and it is still possible to overwrite the defaults
% using explicit options in \includegraphics[width, height, ...]{}
\setkeys{Gin}{width=\maxwidth,height=\maxheight,keepaspectratio}
% Set default figure placement to htbp
\makeatletter
\def\fps@figure{htbp}
\makeatother
\setlength{\emergencystretch}{3em} % prevent overfull lines
\providecommand{\tightlist}{%
  \setlength{\itemsep}{0pt}\setlength{\parskip}{0pt}}
\setcounter{secnumdepth}{-\maxdimen} % remove section numbering
\ifLuaTeX
  \usepackage{selnolig}  % disable illegal ligatures
\fi

\title{Final project report - computational methods}
\author{Amanda Costa Ayres Salmeron}
\date{2022-08-17}

\begin{document}
\maketitle

\hypertarget{functional-analysis-of-the-immune-system-of-children-with-microcephaly-using-serum-cytokines}{%
\section{Functional analysis of the immune system of children with
microcephaly using serum
cytokines}\label{functional-analysis-of-the-immune-system-of-children-with-microcephaly-using-serum-cytokines}}

\hypertarget{introduction}{%
\subsection{Introduction}\label{introduction}}

Despite being an old pathology, it was after the ZIKV epidemic that
microcephaly, characterized by a malformation in the central nervous
system (CNS) in which the head circumference is smaller than the typical
size for sex and age, gained greater social and scientific visibility
(DINIZ, 2016). Between 2015 and 2018, 17,116 cases of microcephalic
children due to congenital infections were reported in Brazil, with
58.3\% of cases concentrated in the Northeast (SECRETARIA DE VIGILÂNCIA
EM SAÚDE, 2020). In the same period, the state of Rio Grande do Norte
had 163 laboratory confirmed cases (SECRETARIA DE SURVEILLANCE IN
HEALTH, 2020). By monitoring these patients in the first years of life,
it was possible to identify a series of symptoms, in addition to the
neurological damage, which is already well established. As congenital
infections normally occur during a critical period of development and
formation of lymphoid organs, they can affect leukocytopoiesis and
erythropoiesis, compromising the immune system of affected children and
leading to different levels of immunosuppression. Previous studies from
our group showed that children with microcephaly have a high number of
hospitalizations and diagnosis of pneumonia, associated with important
changes in the size, topography and volume of lymphoid organs,
generalized leukocytosis with morphological changes in lymphocytes and
neutrophils, allergic susceptibility, in addition to have impaired
memory T cell response.

\hypertarget{objective}{%
\subsection{Objective}\label{objective}}

Knowing that children with microcephaly have morphometric alterations in
the spleen, thymus and cervical lymph nodes, in addition to generalized
leukocytosis with morphological alterations of lymphocytes and
neutrophils, our objective was to analyze the functionality of immune
cells by measuring serum concentrations of IFN-γ cytokines, IL-2, IL-4,
IL-5, IL-6 and TNF-a.

\hypertarget{methods}{%
\subsection{Methods}\label{methods}}

\hypertarget{subjects-and-ethical-considerations}{%
\subsubsection{Subjects and ethical
considerations}\label{subjects-and-ethical-considerations}}

The volunteers of this research were children aged between 9 months and
8 years of age, of both sexes. The children were divided into two
groups: the Control Group and the Microcephaly Group. The CG consisted
of 17 children, aged between 9 months and 8 years, both genders, without
microcephaly or neuromotor dysfunctions and who also do not have
immunocompromise or autoimmune disorders. The GM consisted of 24
children of the same age as the CG, both genders, diagnosed with
microcephaly (head circumference two standard deviations below the mean
for gender and gestational period) and without a diagnosis of secondary
immunodeficiency or autoimmune disorders. The clinical diagnosis of
Congenital Zika Virus Syndrome was performed according to the protocol
established by the Ministry of Health of Brazil (BRASIL, 2017) and as
described by previous studies (DE ARAÚJO et al., 2018; SILVA et al.,
2018). ; MOORE et al., 2017). All participants were recruited from the
Anita Garibaldi Health Education and Research Center (CEPS; Macaíba --
RN) and children who were vaccinated or had suspected infections in the
last 14 days were excluded from the work. The execution of the study
followed the rules of the National Health Council, Resolution
No.~466/2012. The research was approved by the Ethics Committee for
Research with Human Beings on 11.03.17 under protocol No.~CAAE
74871317.8.0000.5292 and 17583419.7.0000.5537. The collections were only
started after completing and signing the free and informed consent form.

\hypertarget{blood-collection}{%
\subsubsection{Blood collection}\label{blood-collection}}

Peripheral blood samples (10 mL) were collected by venipuncture with a
sterile, disposable 10 mL syringe and transferred to a collection tube
containing ethylenediamine tetraacetic acid (EDTA).

\hypertarget{dosage-of-cytokines}{%
\subsubsection{Dosage of cytokines}\label{dosage-of-cytokines}}

The cytokines IFN-γ, IL-2, IL-4, IL-5, IL-6, IL-10 and TNFα were
quantified in serum samples from children of both groups by the
Cytometric Bead Array (CBA) method using kits (Flex Cytometric Bead
Array Enhanced Sensitivity, BD Pharmingen, USA), according to the
manufacturer's instructions. Samples were acquired using the Fortessa
LSR flow cytometer (BD Biosciences, USA), data analysis was performed
using the FCAP Array software (BD Biosciences, USA). The detection
limits were: 14.84 fg/mL for IFN-γ, 88.9 fg/mL for IL-2, 144.4 fg/mL for
IL-4, 67.8 fg/mL for IL-5, 68.4 fg/ml for IL-6, 13.7 fg/ml for IL-10 and
67.3 fg/ml for TNFα.

\hypertarget{data-analysis}{%
\subsubsection{Data analysis}\label{data-analysis}}

The results were organized in an excel spreadsheet and imported into R
studio using the csv format.

Specify requirements in terms of the OS and R libraries Where to find
scripts, functions (if any), reports, raw and processed data, outputs

\hypertarget{results}{%
\subsection{Results}\label{results}}

An increase in the concentrations of IFNγ cytokines (p = 0.0097), IL-4
(p = 0.0223) and IL-2 (p = 0.0185) was observed in children in the
microcephaly group even in the absence of symptomatic infections, or
parasitic and genitourinary infections. asymptomatic. The cytokines
IL-6, IL-5, TNF, IL-10 showed no significant difference between
children in the microcephaly group and the control group.

\end{document}
